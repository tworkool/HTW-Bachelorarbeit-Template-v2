% defined commands and variables

\newcommand{\myparagraph}[1]{\paragraph{#1}\mbox{}\\}

\newcommand{\gloss}[2]{\textbf{#1}\\#2\newparagraph}

\newcommand{\inquote}[1]{
	\glqq{#1}\grqq{}
}

\newcommand{\mkred}[1]{
	\textcolor{red}{#1}
}

\newcommand{\source}[1]{\caption*{ {\small \textbf{Quelle}: {#1}} } }

\newcommand{\quoting}[2]{
	\begin{quote}
		\glqq\textit{#1}\grqq
	\end{quote}
	\begin{flushright}
		\texttwelveudash #2
	\end{flushright}
}

\newcommand{\codeinline}[1]{
	{\bfseries\small\texttt{#1}}
}

% zweite eckige klammer ist optionales Argument #1 und ist standardmäßig frei. 
% Wird benutzt für weitere optionen für \inputminted
\newcommand{\codefull}[5][]{
	\begin{code}
		\captionof{listing}{#3}
		\label{#2}
		\inputminted[breaklines, bgcolor=codeBgColor, linenos, #1]{#4}{#5}
		%		\caption{#2}
	\end{code}
}

%%% LABEL ALWAYS AFTER CAPTION %%%
\newcommand{\image}[4][]{
	\begin{figure}[H]
		%
		\centering
		%\includegraphics[width=1\textwidth]{images/RIFB.png}
		%\includegraphics[height=10cm, keepaspectratio=true]{#1}
		\includegraphics[width=\ifthenelse{\equal{#1}{}}{1\textwidth}{#1}, keepaspectratio=true]{#2}
		\caption{#4}
		\label{#3}
	\end{figure}
}

\newcommand{\doublefigure}[6]{
	\begin{figure}[H]
		\centering
		\begin{subfigure}[b]{.5\textwidth}
			\centering
			%\includegraphics[width=0.9\textwidth]{#1}
			\includegraphics[height=10cm]{#1}
			\caption{#2}
		\end{subfigure}%
		\begin{subfigure}[b]{.5\textwidth}
			\centering
			%\includegraphics[width=0.9\textwidth]{#3}
			\includegraphics[height=10cm]{#3}
			\caption{#4}
		\end{subfigure}
		\caption{#5}
		\label{#6}
	\end{figure}
}

\newcommand{\newparagraph}{\\[12pt]}